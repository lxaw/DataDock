\documentclass[conference]{IEEEtran}
\IEEEoverridecommandlockouts
% The preceding line is only needed to identify funding in the first footnote. If that is unneeded, please comment it out.
\usepackage{cite}
\usepackage{amsmath,amssymb,amsfonts}
\usepackage{algorithmic}
\usepackage{graphicx}
\usepackage{textcomp}
\usepackage{xcolor}
% Hrefs
\usepackage{hyperref}
\def\BibTeX{{\rm B\kern-.05em{\sc i\kern-.025em b}\kern-.08em
    T\kern-.1667em\lower.7ex\hbox{E}\kern-.125emX}}
\begin{document}

\title{
    Delta: An Open Source Data Service for Research
}

\author{\IEEEauthorblockN{Lexington Whalen}
\IEEEauthorblockA{\textit{Department of Computer Science} \\
\textit{University of South Carolina}\\
Columbia, United States\\
LAWHALEN@email.sc.edu}
\and
\IEEEauthorblockN{Homayoun Valafar}
\IEEEauthorblockA{\textit{Department of Computer Science} \\
\textit{University of South Carolina}\\
Columbia, United States\\
homayoun@cec.sc.edu}
}

\maketitle

\begin{abstract}
Every research project necessitates data, often requiring sharing and collaborative review within a team. 
Traditionally, services like Dropbox \cite{dropbox}, Google Drive \cite{googledrive}, or OneDrive \cite{onedrive}  
have been employed for storing research data. However, the limitations of these platforms become evident with the growing 
scale of collected data. Existing file-sharing services generally mandate paid subscriptions for increased storage or additional members, 
diverting research funds from addressing the core research problem that a lab is attempting to work on. 
Moreover, these services often lack direct features for reviewing or 
commenting on data quality, a vital part of ensuring high quality data generation. 
In response to these challenges, we present Delta, a specialized file transfer service crafted for specifically for researchers. 
Delta operates as an application hosted on a research lab server. This design ensures that, with access to a machine and an internet connection, 
teams can facilitate file storage, transfer, and review without incurring extra costs. Being an open-source project, Delta can be customized to 
suit the unique requirements of any research team, and is able to evolve to meet the needs of the research community. We open source the code here: 
\href{https://github.com/lxaw/Delta}{https://github.com/lxaw/Delta}.

\end{abstract}

\begin{IEEEkeywords}
data transfer, data storage, data reviewing, data sharing
\end{IEEEkeywords}

\section{Introduction}
There is a noticeable dearth of quality open source research services with regards to data.
Most data services are merely file sharing ones, that only allow file upload, download, and organization.
While this does allow researchers to transmit their collected data among another, 
it lacks many critical features that most researchers would like to have, such as 
quality control, commenting, tagging, organization under groups, annotation, and more.
Furthermore, these services are oftentimes prohibitively expensive to smaller research labs, and their
free versions have limitations on both number of users and amount of storage. 
The current state of research is to use services such as Microsoft's OneDrive \cite{onedrive}, Dropbox \cite{dropbox},
or Google Drive \cite{googledrive}. Each of these services 
[something about service limitations]
\\\\
Furthermore, the modern times have seen unprecedented growth in the amount of data collected.
[something about deep learning data, internet of things, personal data growth]

\section{The Problem}
[write about the architecture]

\subsection{Current Alternatives}
[write about the rest apis]

\section{Issues with Current Approaches}
[write about the issues with the current approaches]

\section{Delta Architecture}
[write about delta architecture]

\subsection{Backend}

\subsection{Frontend}

\section{Current Product}

\section{Future Goals}


\section*{References}

Please number citations consecutively within brackets \cite{b1}. The 
sentence punctuation follows the bracket \cite{b2}. Refer simply to the reference 
number, as in \cite{b3}---do not use ``Ref. \cite{b3}'' or ``reference \cite{b3}'' except at 
the beginning of a sentence: ``Reference \cite{b3} was the first $\ldots$''

Number footnotes separately in superscripts. Place the actual footnote at 
the bottom of the column in which it was cited. Do not put footnotes in the 
abstract or reference list. Use letters for table footnotes.

Unless there are six authors or more give all authors' names; do not use 
``et al.''. Papers that have not been published, even if they have been 
submitted for publication, should be cited as ``unpublished'' \cite{b4}. Papers 
that have been accepted for publication should be cited as ``in press'' \cite{b5}. 
Capitalize only the first word in a paper title, except for proper nouns and 
element symbols.

For papers published in translation journals, please give the English 
citation first, followed by the original foreign-language citation \cite{b6}.

\begin{thebibliography}{00}
\bibitem{googledrive} Google, “Google Drive: Free Cloud Storage for Personal Use,” Google.com, 2019. https://www.google.com/drive/
\bibitem{dropbox} Dropbox, “Dropbox,” Dropbox, 2018. https://www.dropbox.com/
\bibitem{onedrive} “Personal Cloud Storage – Microsoft OneDrive,” Microsoft.com, 2024. https://www.microsoft.com/en-us/microsoft-365/onedrive/
\bibitem{b4} K. Elissa, ``Title of paper if known,'' unpublished.
\bibitem{b5} R. Nicole, ``Title of paper with only first word capitalized,'' J. Name Stand. Abbrev., in press.
\bibitem{b6} Y. Yorozu, M. Hirano, K. Oka, and Y. Tagawa, ``Electron spectroscopy studies on magneto-optical media and plastic substrate interface,'' IEEE Transl. J. Magn. Japan, vol. 2, pp. 740--741, August 1987 [Digests 9th Annual Conf. Magnetics Japan, p. 301, 1982].
\bibitem{b7} M. Young, The Technical Writer's Handbook. Mill Valley, CA: University Science, 1989.
\end{thebibliography}
\vspace{12pt}

\end{document}
