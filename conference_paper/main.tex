\documentclass[conference]{IEEEtran}
\IEEEoverridecommandlockouts
% The preceding line is only needed to identify funding in the first footnote. If that is unneeded, please comment it out.
\usepackage{cite}
\usepackage{amsmath,amssymb,amsfonts}
\usepackage{algorithmic}
\usepackage{graphicx}
\usepackage{textcomp}
\usepackage{xcolor}
% Hrefs
\usepackage{hyperref}
\def\BibTeX{{\rm B\kern-.05em{\sc i\kern-.025em b}\kern-.08em
    T\kern-.1667em\lower.7ex\hbox{E}\kern-.125emX}}
\begin{document}

\title{
    Delta: An Open Source Data Service for Research
}

\author{\IEEEauthorblockN{Lexington Whalen}
\IEEEauthorblockA{\textit{Department of Computer Science} \\
\textit{University of South Carolina}\\
Columbia, United States\\
LAWHALEN@email.sc.edu}
\and
\IEEEauthorblockN{Homayoun Valafar}
\IEEEauthorblockA{\textit{Department of Computer Science} \\
\textit{University of South Carolina}\\
Columbia, United States\\
homayoun@cec.sc.edu}
}

\maketitle

\begin{abstract}
Every research project necessitates data, often requiring sharing and collaborative review within a team. 
Traditionally, services like Dropbox \cite{dropbox}, Google Drive \cite{googledrive}, or OneDrive \cite{onedrive}  
have been employed for storing research data. However, the limitations of these platforms become evident with the growing 
scale of collected data. Existing file-sharing services generally mandate paid subscriptions for increased storage or additional members, 
diverting research funds from addressing the core research problem that a lab is attempting to work on. 
Moreover, these services often lack direct features for reviewing or 
commenting on data quality, a vital part of ensuring high quality data generation. 
In response to these challenges, we present Delta, a specialized file transfer service crafted for specifically for researchers. 
Delta operates as an application hosted on a research lab server. This design ensures that, with access to a machine and an internet connection, 
teams can facilitate file storage, transfer, and review without incurring extra costs. Being an open-source project, Delta can be customized to 
suit the unique requirements of any research team, and is able to evolve to meet the needs of the research community. We open source the code here: 
\href{https://github.com/lxaw/Delta}{https://github.com/lxaw/Delta}.

\end{abstract}

\begin{IEEEkeywords}
data transfer, data storage, data reviewing, data sharing
\end{IEEEkeywords}

\section{Introduction}
There is a noticeable dearth of quality open source research services with regards to data.
Most data services are merely file sharing ones, that only allow file upload, download, and organization.
While this does allow researchers to transmit their collected data among another,
it lacks many critical features that most researchers would like to have, such as
quality control, commenting, tagging, organization under groups, annotation, and more.
Furthermore, these services are oftentimes prohibitively expensive to smaller research labs, and their
free versions have limitations on both number of users and amount of storage.
The current state of research is to use services such as Microsoft's OneDrive \cite{onedrive}, Dropbox \cite{dropbox},
or Google Drive \cite{googledrive}. Each of these services have limitations in terms of storage capacity, user management,
and feature set that make them suboptimal for research data management \cite{kowalczyk2018data}.
\\
Furthermore, the modern times have seen unprecedented growth in the amount of data collected.
The era of big data has brought about massive datasets, particularly in fields like genomics, astronomy, and social media analytics \cite{stephens2015big}.
Machine learning and deep learning have further amplified the need for large, high-quality datasets for training models \cite{sun2017revisiting}.
The Internet of Things (IoT) is another significant contributor to the data deluge, with billions of connected devices generating continuous streams of data \cite{khan2018iot}.
Even at a personal level, the proliferation of smartphones and digital services has led to an explosion of user-generated data \cite{reinsel2018digitization}.
\\
This rapid growth in data volume and variety has outpaced the capabilities of traditional data management tools and practices.
Researchers often struggle with the challenges of storing, organizing, sharing, and collaborating on large datasets \cite{wallis2013if}.
The lack of specialized tools for research data management leads to ad-hoc solutions, data silos, and inefficiencies in the research process \cite{wilkinson2016fair}.
\\
To address these challenges, we present Delta, an open source data service designed specifically for the needs of researchers.
Delta provides a platform for efficient data storage, sharing, and collaboration, with features tailored to the research workflow.
By leveraging open source technologies and a modular architecture, Delta enables customization and extensibility to meet the diverse requirements of different research domains.
The open source nature of Delta also promotes transparency, reproducibility, and community-driven development \cite{gezelter2015open}.

\section{The Problem}
In the era of data-driven research, scientists across various disciplines are grappling with an unprecedented influx of data. From high-throughput sequencing in genomics \cite{stephens2015big} to large-scale simulations in physics \cite{vogelsberger2014introducing}, researchers are generating and collecting massive datasets at an astounding rate. However, managing this data effectively poses significant challenges that can hinder research productivity and collaboration.

One of the primary issues researchers face is the need for efficient data labeling and annotation. Raw data often requires contextualization and metadata to be meaningful and usable for analysis \cite{wilkinson2016fair}. Manually annotating large datasets is time-consuming and prone to inconsistencies, especially when multiple researchers are involved. The lack of standardized annotation frameworks and tools further compounds this problem, leading to a fragmented and inefficient data labeling process.

Quality control is another critical concern in research data management. Ensuring the accuracy, completeness, and reliability of data is essential for drawing valid conclusions and reproducing results \cite{eisner2018data}. However, reviewing and validating large datasets manually is a daunting task, particularly in collaborative projects where data is collected by multiple individuals or teams. The absence of systematic quality control mechanisms can lead to errors, inconsistencies, and a lack of trust in the data.

Effective organization and management of research data is also a significant challenge, both at the individual researcher level and within research groups. Principal Investigators (PIs) need to keep track of various datasets, their versions, and associated metadata, often relying on ad-hoc solutions like spreadsheets or file naming conventions \cite{kowalczyk2018data}. At the group level, data silos can emerge when different teams use incompatible data formats or storage systems, hindering collaboration and data integration.

Sharing data between research groups is another pain point in the current research landscape. Collaborative projects often involve multiple institutions and disciplines, each with their own data management practices and infrastructure \cite{tenopir2011data}. Incompatible data formats, lack of standardization, and concerns about data ownership and attribution can impede smooth data sharing and reuse. Moreover, transferring large datasets across institutional boundaries can be time-consuming and subject to security and privacy constraints.

Addressing these challenges requires a comprehensive and integrated approach to research data management. Researchers need tools and platforms that streamline data annotation, enable robust quality control, facilitate organization and discovery, and promote seamless data sharing and collaboration. Existing solutions, such as generic cloud storage services or ad-hoc scripts, often fall short in terms of functionality, scalability, and ease of use. There is a clear need for a purpose-built, open-source data management platform that caters to the unique requirements of scientific research.

\section{Current Alternatives}
Several cloud storage providers offer APIs and developer tools that enable programmatic access to their services. For example:

\begin{itemize}
\item Google Drive provides a REST API \cite{googledrive} for uploading, downloading, searching, and manipulating files stored on Google Drive. Client libraries are available in multiple programming languages.

\item Dropbox offers a similar REST API \cite{dropbox} with SDKs for major platforms to integrate Dropbox capabilities into applications. Key features include file upload/download, sharing, search, and user management.

\item Microsoft OneDrive also has a comprehensive REST API \cite{onedrive} for accessing OneDrive files, folders, and other data. Client libraries support integration with web, mobile, and desktop apps.

\item Box provides a content management platform with a REST API \cite{box} for building custom applications. It offers features like file preview, version control, and granular access permissions, catering to enterprise needs.

\item Amazon S3 (Simple Storage Service) is a scalable object storage service with an API \cite{amazonweb} for storing and retrieving data. While not primarily designed for end-user file management, it's often used as a foundation for building cloud storage applications.
\end{itemize}

While these APIs enable building custom applications with cloud storage, the underlying limitations of the services still apply - costs scale with storage and user needs, and specialized features for research data management are lacking. Leveraging the APIs still requires significant development effort to create a tailored solution. Furthermore, some services like Amazon S3 are more suited to developers and lack user-friendly interfaces out-of-the-box.
Furthermore, these services are not open sourced, so any modifications (i.e. addition of service, design change, bug fix) cannot be done directly by the users.
We make such issues more clear in $\textit{Issues with Current Approaches}$.

\section{Issues with Current Approaches}
While current cloud storage services offer APIs and enable building custom applications, they have several notable shortcomings, especially in the context of research data management:

\begin{enumerate}
\item \textbf{Cost:} Services like Google Drive, Dropbox, and OneDrive can become expensive as data storage needs grow, often requiring paid subscriptions for additional storage or users. This diverts funds from core research activities.

\item \textbf{Lack of Specialization:} General-purpose storage services lack features tailored for research, such as data quality control, peer review workflows, metadata management, and data provenance tracking.

\item \textbf{Limited Customization:} Although APIs enable custom app development, the underlying platforms cannot be easily modified or extended. Researchers cannot add new features or modify existing behavior to suit their specific needs.

\item \textbf{Vendor Lock-In:} By relying on proprietary services, researchers risk being locked into a particular vendor's ecosystem. Migration to alternative platforms can be difficult and costly.

\item \textbf{Data Ownership and Control:} With commercial services, there may be ambiguity around data ownership and control. Researchers may have concerns about intellectual property rights and the ability to access their data if a service is discontinued.

\item \textbf{Data Privacy and Security:} Storing sensitive research data on third-party servers raises privacy and security concerns. Researchers may be hesitant to entrust confidential data to external providers.

\item \textbf{Collaboration Barriers:} While cloud services facilitate file sharing, they often lack advanced collaboration features like real-time co-authoring, version control, and granular access controls that are vital for research teams.

\item \textbf{Integration Challenges:} Integrating cloud storage with existing research tools and workflows can be challenging. Researchers may need to develop custom glue code or rely on limited third-party integrations.

\item \textbf{Dependency on Internet Connectivity:} Cloud services require reliable internet access, which can be a constraint in field research settings or areas with limited connectivity.

\item \textbf{Long-Term Preservation:} Commercial services may not prioritize long-term data preservation, which is crucial for research reproducibility and data archiving. There may be uncertainties about data durability and accessibility over extended periods.
\end{enumerate}

An open source solution like Delta addresses these issues by providing a specialized, customizable, and cost-effective platform for research data management. By hosting Delta on their own infrastructure, research teams have full control over their data, can tailor the platform to their specific needs, and avoid vendor lock-in. The open source nature ensures transparency, enables community-driven development, and allows for integration with a wide range of tools. Moreover, an open source solution can be deployed in local or offline environments, mitigating concerns about internet connectivity and data privacy. Delta empowers researchers to manage their data on their own terms, prioritizing the unique requirements of scientific research.

\section{Delta Architecture}
[write about delta architecture]

\subsection{Backend}

\subsection{Frontend}

\section{Current Product}

\section{Future Goals}


\section*{References}

Please number citations consecutively within brackets \cite{b1}. The 
sentence punctuation follows the bracket \cite{b2}. Refer simply to the reference 
number, as in \cite{b3}---do not use ``Ref. \cite{b3}'' or ``reference \cite{b3}'' except at 
the beginning of a sentence: ``Reference \cite{b3} was the first $\ldots$''

Number footnotes separately in superscripts. Place the actual footnote at 
the bottom of the column in which it was cited. Do not put footnotes in the 
abstract or reference list. Use letters for table footnotes.

Unless there are six authors or more give all authors' names; do not use 
``et al.''. Papers that have not been published, even if they have been 
submitted for publication, should be cited as ``unpublished'' \cite{b4}. Papers 
that have been accepted for publication should be cited as ``in press'' \cite{b5}. 
Capitalize only the first word in a paper title, except for proper nouns and 
element symbols.

For papers published in translation journals, please give the English 
citation first, followed by the original foreign-language citation \cite{b6}.

\begin{thebibliography}{00}
\bibitem{googledrive} Google, “Google Drive: Free Cloud Storage for Personal Use,” Google.com, 2019. https://www.google.com/drive/
\bibitem{dropbox} Dropbox, “Dropbox,” Dropbox, 2018. https://www.dropbox.com/
\bibitem{onedrive} “Personal Cloud Storage – Microsoft OneDrive,” Microsoft.com, 2024. https://www.microsoft.com/en-us/microsoft-365/onedrive/
\bibitem{box} Box, "Box Developer Documentation," Box, 2023. https://developer.box.com/
\bibitem{amazonweb} Amazon Web Services, "Amazon S3 API Reference," Amazon.com, 2023. https://docs.aws.amazon.com/AmazonS3/latest/API/Welcome.html
\bibitem{kowalczyk2018data} D. R. Kowalczyk and S. Y. Shankar, "Data sharing in the sciences," \textit{Annual Review of Information Science and Technology}, vol. 45, no. 1, pp. 247-294, 2011, doi: 10.1002/aris.2011.1440450113.
\bibitem{stephens2015big} Z. D. Stephens et al., "Big Data: Astronomical or Genomical?" \textit{PLOS Biology}, vol. 13, no. 7, p. e1002195, Jul. 2015, doi: 10.1371/journal.pbio.1002195.
\bibitem{sun2017revisiting} C. Sun, A. Shrivastava, S. Singh, and A. Gupta, "Revisiting Unreasonable Effectiveness of Data in Deep Learning Era," in \textit{2017 IEEE International Conference on Computer Vision (ICCV)}, Oct. 2017, pp. 843-852, doi: 10.1109/ICCV.2017.97.
\bibitem{khan2018iot} M. A. Khan, K. Salah, "IoT security: Review, blockchain solutions, and open challenges," \textit{Future Generation Computer Systems}, vol. 82, pp. 395-411, 2018, doi: 10.1016/j.future.2017.11.022.
\bibitem{reinsel2018digitization} D. Reinsel, J. Gantz, and J. Rydning, "The Digitization of the World - From Edge to Core," \textit{IDC White Paper}, Nov. 2018. [Online]. Available: https://www.seagate.com/files/www-content/our-story/trends/files/idc-seagate-dataage-whitepaper.pdf
\bibitem{wallis2013if} J. C. Wallis, E. Rolando, and C. L. Borgman, "If We Share Data, Will Anyone Use Them? Data Sharing and Reuse in the Long Tail of Science and Technology," \textit{PLOS ONE}, vol. 8, no. 7, p. e67332, Jul. 2013, doi: 10.1371/journal.pone.0067332.
\bibitem{wilkinson2016fair} M. D. Wilkinson et al., "The FAIR Guiding Principles for scientific data management and stewardship," \textit{Scientific Data}, vol. 3, no. 1, Art. no. 1, Mar. 2016, doi: 10.1038/sdata.2016.18.
\bibitem{gezelter2015open} J. D. Gezelter, "Open Source and Open Data Should Be Standard Practices," \textit{The Journal of Physical Chemistry Letters}, vol. 6, no. 7, pp. 1168-1169, Apr. 2015, doi: 10.1021/acs.jpclett.5b00285.
\bibitem{stephens2015big} Z. D. Stephens et al., "Big Data: Astronomical or Genomical?" \textit{PLOS Biology}, vol. 13, no. 7, p. e1002195, Jul. 2015, doi: 10.1371/journal.pbio.1002195.
\bibitem{vogelsberger2014introducing} M. Vogelsberger et al., "Introducing the Illustris Project: Simulating the coevolution of dark and visible matter in the Universe," \textit{Monthly Notices of the Royal Astronomical Society}, vol. 444, no. 2, pp. 1518-1547, Oct. 2014, doi: 10.1093/mnras/stu1536.
\bibitem{wilkinson2016fair} M. D. Wilkinson et al., "The FAIR Guiding Principles for scientific data management and stewardship," \textit{Scientific Data}, vol. 3, no. 1, Art. no. 1, Mar. 2016, doi: 10.1038/sdata.2016.18.
\bibitem{eisner2018data} D. A. Eisner, "Reproducibility of science: Fraud, impact factors and carelessness," \textit{Journal of Molecular and Cellular Cardiology}, vol. 114, pp. 364-368, Jan. 2018, doi: 10.1016/j.yjmcc.2017.10.009.
\bibitem{kowalczyk2018data} D. R. Kowalczyk and S. Y. Shankar, "Data sharing in the sciences," \textit{Annual Review of Information Science and Technology}, vol. 45, no. 1, pp. 247-294, 2011, doi: 10.1002/aris.2011.1440450113.
\bibitem{tenopir2011data} C. Tenopir et al., "Data Sharing by Scientists: Practices and Perceptions," \textit{PLOS ONE}, vol. 6, no. 6, p. e21101, Jun. 2011, doi: 10.1371/journal.pone.0021101.
\end{thebibliography}
\vspace{12pt}

\end{document}
